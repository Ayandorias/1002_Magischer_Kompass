\documentclass[10pt,a4paper,twocolumn,openany]{book}
%\documentclass[12pt,a4paper,openany]{book}
\usepackage[bg-full]{lib/rpg-book}
\usepackage{lipsum}
%\usepackage{makeidx}
\usepackage{glossaries}
\usepackage[toc,page]{appendix}
\usepackage{booktabs}
\RequirePackage{titletoc}
\usepackage[ngerman]{babel}
\usepackage{wrapfig}
\usepackage{imakeidx}
\usepackage{etoolbox}

%\usepackage[frame,width=216truemm,height=303truemm,center]{crop}/home/heresy/Privat/Projekte/Rollenspiel - Nebenwelten/Publikationen/0001_Quellenbände/999_Charaktertagebuch/lib

\usepackage{tikz}
\usetikzlibrary{mindmap, trees}


\makeindex

\renewcommand{\chaptername}{Kapitel}
\renewcommand{\contentsname}{Inhaltsverzeichnis}
\setlength{\headheight}{15pt}
\addtolength{\topmargin}{-15pt}


% Start document
\begin{document}
	
\input{Misc/version.tex}
\frontmatter


\rpgMakeCover[
image = Bilder/cover,
%logo = ,%Regelwerk/Bilder/Logo/logo,
title = "", %Kernregelwerk,
subtitle =
]

\cleardoublepage

{
	\onecolumn
	%\thispagestyle{empty}
	%Nun stellen wir eine kleinere Schriftgröße ein, setzen den Absatzeinzug auf Null und dafür den Absatzabstand auf eine Leerzeile. Das ermöglicht es uns, die verschiedenen Blöcke des Impressums einfach als Absätze zu schreiben, und wir müssen nicht jedesmal einen manuellen Abstand dazwischen einfügen.
	
	\footnotesize
	\setlength{\parindent}{0pt}
	\setlength{\parskip}{\baselineskip}
%	Das Impressum soll am unteren Seitenende ausgerichtet werden, allerdings können wir dieses Mal nicht \vfill benutzen, da es als Abkürzung für \vspace{\fill} am Seitenbeginn ignoriert würde.

	
	\vspace*{1cm}
	
	%Die Angaben setzen wir in Blöcken/Absätzen, deren Zeilen wir mit \\ trennen.
	\begin{center}
	\fontsize{50}{50}\steamfont{Impresum}	
	
	\vspace*{2cm}
	
	\steamfont{\Huge Author}\\
	Bruno Pierucki
	
	\vspace*{1cm}
	
	\steamfont{\Huge Covermotiv}\\
	KI gneriert
	
	\vspace*{1cm}
	
	\steamfont{\Huge Innenillustrationen}\\
	KI gneriert
	
	\vspace*{1cm}
	
	\steamfont{\Huge Satz und Layout}\\
	Bruno Pierucki
	
	\vspace*{1cm}
	
%	\steamfont{\Huge Helfer}\\
%	Carschti, Jörg und Lars	
	\vspace*{\fill}
	
	\input{lib/templates/copyright.tex}	
	
	\end{center}
	
	\twocolumn
}

\tableofcontents

\mainmatter
%\cleardoublepage











**Magisches Regelwerk: Der Kristallklang**

\chapter{Einleitung und Grundprinzipien der Magie}

\section{Der Geburtskristall}

Auf Arcanis, einer Welt, in der die Magie so tief in das Gewebe der Existenz eingewoben ist wie das Licht der Sonnenstrahlen, ist der Geburtskristall das heiligste und persönlichste Geschenk, das ein Mensch bei seiner Ankunft erhält. Er ist kein bloßes Juwel, sondern das pulsierende Herz seiner zukünftigen magischen Fähigkeiten, ein Seelengefährte aus reiner Energie. Diese Kristalle sind so einzigartig wie die Individuen selbst; sie können die unterschiedlichsten, oft wundersamen Formen annehmen – von schillernden geometrischen Gebilden über organisch anmutende, geschwungene Skulpturen bis hin zu kristallinen Blumen, die ewig zu blühen scheinen. Ihre Farben sind ein Spektrum des magischen Potenzials, wobei jede Nuance einen bestimmten Weg andeutet: Ein zartes Pink flüstert von der subtilen Eleganz der Magierkunst, ein tiefes Grün spricht von der erdverbundenen Weisheit der Hexen, und ein leuchtendes Blau kündet von der harmonischen Kraft der Druiden.

Obwohl die Farbe des Kristalls einen vorbestimmten Pfad aufzeigt, ist der freie Wille auf Arcanis ein unantastbares Gut. Jeder Mensch besitzt die Freiheit, seinen eigenen Weg zu wählen, unabhängig von der Vorbestimmung seines Geburtskristalls. Doch der Kristall ist mehr als nur ein Wegweiser; er ist ein Resonator, der es seinem Träger ermöglicht, sich mit bestimmten magischen Strömungen besonders leicht zu verbinden. Wer sich dem Pfad seiner Kristallfarbe anschließt, wird feststellen, dass ihm die entsprechenden Zauber mit einer außergewöhnlichen Leichtigkeit und Meisterschaft von der Hand gehen. Und erst durch diesen untrennbaren Bund mit dem Geburtskristall wird der Mensch überhaupt erst in die Lage versetzt, die Wunder der Magie zu wirken.

\section{Die magische Schöpfung}

Der Geburtskristall ist kein Artefakt, das gesucht oder gefunden werden muss. Er ist ein Akt der reinen magischen Schöpfung, ein Wunder, das sich im heiligsten Moment des menschlichen Lebens entfaltet: der Geburt. In dem Augenblick, in dem ein Kind seinen ersten, lebensbejahenden Schrei in die Welt sendet, manifestiert sich im Geburtsraum ein Kristall, der einzig und allein für dieses Neugeborene bestimmt ist. Es ist, als ob das Universum selbst auf den Ruf des neuen Lebens antwortet und ein Fragment seiner eigenen magischen Essenz als Willkommensgeschenk überreicht. Dieser Kristall ist somit kein Besitz im herkömmlichen Sinne, sondern eine tiefgreifende Verbindung, eine energetische Symbiose, die das Kind von Geburt an begleitet.

Es ist wichtig zu verstehen, dass niemand auf Arcanis dazu gezwungen wird, seine magischen Fähigkeiten zu nutzen oder einem bestimmten magischen Pfad zu folgen. Die Magie ist ein Werkzeug, ein Potenzial, das je nach den individuellen Bedürfnissen und Wünschen eingesetzt werden kann. Manche nutzen ihre innere Magie, um ihre persönlichen Stärken zu entfalten, ihre Sinne zu schärfen oder ihre körperliche Konstitution zu verbessern. Andere integrieren die Magie nahtlos in ihren Alltag, um ihre Arbeit effizienter zu verrichten, sei es als Handwerker, Heiler oder Entdecker. Der Geburtskristall ist somit ein Angebot, kein Zwang, und die Wahl liegt stets beim Individuum.

\section{Die Unzerstörbarkeit der Seele}

Der Geburtskristall ist ein ewiger Begleiter, ein untrennbarer Teil des Menschen, der ihn bei seiner Geburt empfangen hat. Er verweilt während der gesamten Lebenszeit seines Besitzers in dessen Obhut, ein konstantes Leuchtfeuer seiner inneren Magie. Und in einem Reich, in dem Magie oft die Grenzen des Vorstellbaren sprengt, ist der Geburtskristall selbst ein Symbol für die Unzerstörbarkeit des Lebens. Weder die sengende Glut eines Feuers noch die eisige Umarmung des Winters, noch die zerstörerischen Kräfte übernatürlicher Entitäten können ihm etwas anhaben. Er trotzt jedem Angriff, bleibt intakt und unversehrt, ein Spiegelbild der inneren Widerstandsfähigkeit seines Besitzers.

Erst wenn der letzte Atemzug genommen wird, wenn der Lebensfaden reißt und der Besitzer verstirbt, erfüllt sich das Schicksal des Kristalls. In diesem Moment erlischt auch sein inneres Licht, und er zerfällt zu reinem Aether, der wieder in den Kreislauf der magischen Energie auf Arcanis zurückkehrt. Es ist ein poetischer Akt der Freigabe, ein Abschied, der die ewige Verbindung zwischen Leben, Magie und Tod auf dieser Welt unterstreicht.

\section{Das Ritual der Erhebung}

Der Geburtskristall ist kein statisches Objekt; er ist ein lebendiger Behälter des Potenzials, der mit seinen Aufgaben wächst und sich mit der Entwicklung seines Besitzers entfaltet. Um jedoch die volle Pracht und Kraft dieses Potenzials zu entfesseln, ist ein heiliges Ritual notwendig. Dieses Ritual, tief verwurzelt in den magischen Traditionen von Arcanis, wird von einem erfahrenen und geeigneten Magier, einer weisen Hexe oder einem harmonischen Druiden durchgeführt. Es ist ein komplexer Prozess, der die energetische Verbindung zwischen dem Individuum und seinem Kristall vertieft, die darin schlummernden Reserven freisetzt und die Kanäle für die magische Energie erweitert.

Nach Abschluss dieses Rituals ist der Besitzer in der Lage, wesentlich mächtigere und komplexere Zauber zu wirken. Die Grenzen seiner magischen Fähigkeiten verschieben sich dramatisch. Er kann nicht nur stärkere Varianten bekannter Zauber anwenden, sondern auch völlig neue Effekte und Modifikationen in seine vorhandenen Zauber integrieren. Ein einfacher Heilzauber könnte nun mehrere Wunden gleichzeitig schließen, oder ein Feuerball könnte eine größere Fläche verzehren – die Möglichkeiten sind so vielfältig wie die Magie selbst.

\section{Der Atem des Kosmos}

Die magische Energie, die auf Arcanis fließt und pulsiert, ist bekannt als Aether. Sie ist der grundlegende Stoff, aus dem die Wunder dieser Welt gewebt sind, und ihre Quelle ist so ehrfurchtgebietend wie das Leben selbst: die Sonne Lunaris. Vor Äonen, in einer Zeit, die nur noch in den ältesten Legenden existiert, war es Lunaris, die das Leben auf Arcanis generierte. Ihre pink-rosanen Strahlen durchdrangen die urzeitliche Materie und hauchten ihr eine magische Essenz ein, die bis heute die Lebensader des Planeten bildet. Es sind diese unaufhörlichen Strahlen, die die Geburtskristalle der Arcaner fortwährend mit Aether befüllen und sie zu den leuchtenden Gefäßen machen, die sie sind. Die Sonne Lunaris ist somit nicht nur eine Lichtquelle, sondern die ewige Gebende, die unendliche Quelle der Magie.

\section{Der Ozean des Aethers}

Der Aether selbst ist eine unendliche Ressource, ein kosmischer Ozean der Energie, der niemals versiegt. Doch der Einsatz dieser Magie durch einen Geburtskristall ist begrenzt. Jeder Kristall hat eine gewisse Kapazität, wie viel Aether er auf einmal aufnehmen und kanalisieren kann. Es ist wie ein See, der sich aus einem unendlichen Ozean speist, aber nur eine bestimmte Menge Wasser auf einmal fassen kann. Die Menge der verfügbaren Magie in einem Kristall ist direkt proportional zur Anzahl der Rituale, die sein Besitzer durchführt, um seine magische Potenz zu erhöhen. Je häufiger ein solches Ritual vollzogen wird, desto größer wird die Kapazität des Kristalls, Aether zu speichern und zu nutzen, und umso mächtiger kann der Anwender werden.

\section{Die Adern der Welt}

Ja, auf Arcanis durchziehen unsichtbare, aber mächtige Ley-Linien das Land, wie die Adern eines lebendigen Organismus. Diese Linien verstärken die Magie zwar nicht direkt in ihrer rohen Form, aber sie dienen als Katalysatoren, die den Geburtskristallen helfen, den Aether schneller wieder aufzunehmen und ihre Energie zu regenerieren. Sie sind wie Ladestationen, die die Kristalle effizienter aufladen und somit eine schnellere Erholung nach magischen Anstrengungen ermöglichen.

Die Quelle dieser Ley-Linien sind die großen, schwebenden Kristalle, die majestätisch in jedem Land von Arcanis am Himmel schweben. Diese gewaltigen Kristalle sind die Knotenpunkte eines globalen Netzwerks, jeder einzelne ist mit jedem anderen verbunden und bildet ein komplexes Geflecht, das die magische Energie durch die gesamte Welt leitet. Sie sind die Herzstücke des Aether-Netzwerks, die die unendliche Energie von Lunaris durch die Ley-Linien verteilen und so die magische Vitalität von Arcanis aufrechterhalten.

\section{Die eherne Regel der Zeit}

So mächtig und allgegenwärtig die Magie auf Arcanis auch sein mag, sie ist nicht grenzenlos. Es gibt natürliche Beschränkungen für ihre Wirksamkeit, die je nach Art des Zaubers und der Umgebung variieren können. Die Materialbeschaffenheit eines Objekts, die Distanz zum Ziel oder die Komplexität eines beabsichtigten Effekts können die magische Wirksamkeit beeinflussen. So ist es beispielsweise einfacher, einen Holzscheit in Brand zu setzen als einen großen Felsbrocken zu verschieben, und ein Zauber, der über weite Entfernungen gewirkt werden soll, erfordert mehr Konzentration und Aether.

Doch die eine, eherne Regel, die alle magischen Künste auf Arcanis bindet, ist die Unveränderlichkeit der Zeit. Die Manipulation der Zeit ist eine absolute Unmöglichkeit. Kein Zauber, keine arkane Macht, kein mächtiges Ritual kann den Lauf der Zeit verändern, die Vergangenheit rückgängig machen oder die Zukunft vorwegnehmen. Die Zeit ist auf Arcanis eine unantastbare Größe, ein Fluss, der immer nur in eine Richtung fließt, und selbst die größten Magier respektieren diese unüberwindbare Grenze der Magie.


%\clearpage
%
%
%
%
%* **Was ist Magie in dieser Welt?** Eine grundlegende Erklärung, dass Magie eine inhärente Energie ist, die durch Kristalle kanalisiert und manifestiert werden kann.
%
%Als Lunaris Ihre Energie 
%
%* **Der Geburtskristall:** Detaillierte Beschreibung der Funktion und Bedeutung des Kristalls.
%
%Der Kristall ist das wichtigste Stück, was ein Mensch auf Arcanis bei seiner Geburt erhält. Der Kristall kann die unterschiedlichsten Formen annehemn und auch unterschiedliche Farben.
%Je nach Farbe des Kristalls wird der Werdegang des Kindes vorbetimmt. Pink wird Magier, Grün wird Hexe und Blau wird Druide. Jeder Mensch hat dennoch die Entscheidung welchen Weg er letztendlich geht. 
%Allerdings kriegt er Zauber aus diesem Bereich besser hin. 
%Erst durch diesen Kristall ist der Mensch in der Lage Magie zu wirken.
%
%
%* Wie wird er bei der Geburt "gewonnen" oder "verliehen"?
%
%Bei der Geburt eines Kindes entsteht im Raum der Geburt der Krstall, der für das Kind bestimmt ist. Der Kristall entsteht erst dann, wenn das Kind seinen ersten Schrei von sich gibt.
%
%Kein Mensch ist gezweungen seine Magie auch auszuübern, oder irgeneinen magischen Weg einzuschlagen. Manche nutzen die Magie um sich selbst zu stärken, oder die Magie wird genutzt, damit er seine Arbeit besser verrichten kann.
%
%* Ist er permanent oder kann er beschädigt/verloren werden? Was sind die Konsequenzen?
%
%Der Kristall bleibt wärend der kompletten Lebenszeit des Menschen in seinen Besitz. Der Kristall selbst ist unzerstörbar. Weder Feuer noch Eis oder irgendwelche übernatürlichen Kräfte können Ihn zerstören. 
%Erst nachdem der Besitzer verstirbt, stirbt auch der Kristall und gibt seine Macht wieder frei.
%
%* Wie beeinflusst der Kristall die magische Potenz eines Individuums?
%
%Der Kristall wächst mit seinen Aufgaben. Hierzu ist aber ein entsprechendes Ritual notwendig. Dieses wird durch einen geeigneten Magier, Hexe oder Druide durchgeführt.
%Nach dem Ritual ist der BEsitzer in der Lage mächtigere Zauber zu sprechen, oder andere Effekte bei seinen vorhandenen Zaubern zu generieren.
%
%* **Magische Energie (Mana, Äther, etc.):** Wie wird die magische Energie genannt und woher stammt sie?
%
%Die magische Energie wird auch Aether genannt. Diese stammt letztendlich von der Sonne Lunaris. Diese hat zu BEginn das Laben auf Arcanis generiert. Durch Ihre Strahlen, werden die Kristalle immer wieder mit Energie befüllt.
%
%* Ist sie endlich oder unendlich?
%
%Allein die Magie ist unendlich. Der Einsatz der Magie durch einen Kristall ist jedoch begrenzt. Je häufiger ein Ritual durchgeführt wird, umso höher ist die zu Nutzende Magie.
%
%* Gibt es Quellen, die die magische Energie beeinflussen (z.B. Ley-Linien, arkane Anomalien, planetare Konstellationen)?
%
%Es gibt bestimmt Ley-Linien auf Arcanis, die die Magie zwar nicht verstärken, aber dafür sorgen, dass der Kristall schneller wieder an Kraft gewinnt. Es gibt in jedem Land ein großer schwebender Kristall.
%Von jedem dieser kristalle gehen die Ley-Linien aus. Jeder Kristall ist mit jedem anderen verbunden.
%
%* **Grenzen der Magie:** Gibt es natürliche Beschränkungen für die Wirksamkeit der Magie (z.B. Material, Distanz, Komplexität)?
%
%Ja es gibt natürliche Grenzen. Diese können je nach Zauber unterscxhiedlich ausfallen. Was aber auf keinen Fall geht, ist die Zeit zu manipulieren.



\chapter{Magieanwendung für Jedermann}

\section{Die ersten Funken des Aethers}

Wenn ein Arcaner seinen Geburtskristall in Händen hält, beginnt seine Reise in die Welt der Magie. Zunächst befindet sich der Kristall auf der magischen Stufe 1, die den Grundstein für alle weiteren Entwicklungen legt. Auf dieser anfänglichen Stufe sind dem Träger des Kristalls nur Zauber zugänglich, die ebenfalls dieser Stufe entsprechen. Es ist wie das Erlernen einer neuen Sprache; man beginnt mit den einfachsten Vokabeln und Sätzen, bevor man komplexe Gedankengebäude errichten kann. Jeder, der einen Geburtskristall besitzt, trägt das Potenzial in sich, grundlegende magische Effekte zu manifestieren, doch diese müssen, selbst auf der ersten Stufe, noch bewusst erlernt werden. Das Entzünden eines kleinen, schwebenden Lichts, das kurze Schwebenlassen eines leichten Gegenstandes oder die Ausführung einfacher Reparaturen an Alltagsgegenständen sind Beispiele für solche elementaren Fertigkeiten, die mit der Zeit und Übung erworben werden können. Sie sind die ersten, zarten Funken des Aethers, die von dem noch jungen Kristall freigesetzt werden.

\section{Der Preis der Macht}

Die Anwendung von Magie ist kein unbegrenzter Quell der Kraft; sie fordert ihren Tribut vom Anwender. **Jeder, der Magie wirkt,** verliert etwas von seinem Aether-Vorrat des Geburtskristalls und somit auch an der eigenen Energie. Ähnlich einem Muskel, der nach Anstrengung Erholung benötigt, führt die Magieanwendung zu einer spürbaren Erschöpfung. Dies kann sich als mentale Trägheit, körperliche Müdigkeit oder ein allgemeines Gefühl der Entleerung äußern. Die Regeneration des verlorenen Aethers ist ein natürlicher Prozess, der vor allem durch Ruhe und Schlaf gefördert wird. Während dieser Ruhephasen werden die Kristalle durch die allgegenwärtigen Ley-Linien auf Arcanis passiv wieder mit Aether aufgefüllt. Es ist ein stilles, kontinuierliches Strömen, das die Kristalle langsam, aber stetig revitalisiert, bis sie wieder bereit sind für neue magische Herausforderungen.

\section{Das Fundament der Zauberkunst}

Im Gegensatz zu manch anderem Reich, wo grundlegende magische Fähigkeiten angeboren sind, gibt es auf Arcanis keinen "Basiszauber", den jeder von vornherein beherrschen kann. Jede Form der Zauberkunst, selbst die vermeintlich einfachste, muss sorgfältig erlernt und geübt werden. Dies erfordert die Anleitung eines Lehrmeisters, der die Geheimnisse des Aethers zu entschlüsseln vermag. Ob es sich um den weisen Lehrmeister in den ehrwürdigen Hallen der Akademie handelt, die geheimnisvolle Hexe, die tief im verwunschenen Wald ihre Künste lehrt, der verschrobene Druide, der unermüdlich Kräuter sammelt und das Flüstern der Natur versteht, oder eben einfach nur dein Elternteil, das lehrt, was man tun muss, um den Zauber zu wirken – das Wissen wird von Mund zu Mund, von Geist zu Geist weitergegeben. Das Erlernen eines Zaubers ist ein Prozess der Nachahmung, des Verständnisses der zugrunde liegenden Formeln und des Eintauchens in die subtilen Nuancen der Aether-Manipulation. Erst durch diese sorgfältige Schulung wird der Weg frei für die Entfaltung der individuellen magischen Potenziale.

\section{Die Symphonie von Aether und Zahnrädern}

Die reine, unvermittelte Magie, die aus den Geburtskristallen der Arcaner strömt, hat keine direkte Einwirkung auf die Steampunk-Technologie. Die filigranen Zahnräder, komplexen Dampfmotoren und raffinierten Mechanismen dieser Welt funktionieren nach ihren eigenen, physikalischen Gesetzen. Doch das bedeutet nicht, dass Aether und Technologie voneinander getrennt sind. Vielmehr können sie auf faszinierende Weise miteinander kombiniert werden, um etwas gänzlich Neues und Außergewöhnliches zu schaffen.

Auf Arcanis existieren weitere, besondere Kristalle, die die Fähigkeit besitzen, reine Energie zu speichern – eine Energie, die nicht direkt für arkanen Zauber genutzt werden kann, aber von immensem Wert für die Technologie ist. Jeder, der Zugang zu diesen Energiequellen hat, kann diese Kristalle anzapfen, um Maschinen anzutreiben. Die findigen Wissenschaftler von Arcanis haben gelernt, diese Kristalle geschickt in ihre Konstruktionen zu integrieren und sie als leistungsstarke Energiequellen zu nutzen. Das Faszinierende daran ist, dass diese speziellen Energie-Kristalle nicht nur Energie abgeben, sondern auch durch den Betrieb von Maschinen wieder aufgeladen werden können. Sie sind quasi die Batterien dieser magisch-technologischen Welt.

Allerdings birgt die Nutzung dieser Energie-Kristalle auch eine Gefahr: Wenn sie zu lange nicht aufgeladen werden, verlieren sie ihre gespeicherte Energie vollständig und zerfallen zu Staub. Es ist ein ständiger Kreislauf des Verbrauchs und der Regeneration. Der Füllstand dieser Kristalle wird durch ihre Farbe angezeigt, ähnlich einem Regenbogen, der die Fülle der Energie widerspiegelt. Ein gleißend weißer Kristall ist vollständig gefüllt und pulsiert vor Energie. Von dort aus nimmt die Farbe graduell ab: über Rot, Orange, Gelb, Grün, Blau, Indigo, Violett, bis hin zu einem tiefen Schwarz, der einen vollständig leeren Kristall darstellt. Ist der Kristall schwarz, verbleibt dem Nutzer nur ein einziger Tag, um ihn wieder aufzuladen. Gelingt dies nicht, zerfällt der Kristall unweigerlich zu Staub, und seine gespeicherte Energie ist für immer verloren.

\section{Die Eigenständigkeit des Aethers}

Trotz aller Innovationen und dem Streben nach Fortschritt sind Maschinen auf Arcanis nicht in der Lage, die Magie selbst zu verstärken oder auf eine Art und Weise zu kanalisieren, wie es ein Lebewesen mit seinem Geburtskristall vermag. Die magische Essenz, der Aether, ist ein organisches Phänomen, das untrennbar mit der Lebensenergie und dem Bewusstsein des Trägers verbunden ist. Maschinen können Energie-Kristalle nutzen, aber sie können nicht die subtilen Frequenzen des Aethers manipulieren oder die Resonanz, die zwischen einem Zauberer und seinem Geburtskristall besteht, nachahmen. Die Magie bleibt somit eine Domäne der Lebewesen, während die Technologie die Welt der Mechanik und Energie beherrscht.

\section{Die Grenzen der personalisierten Magie}

Die Verschmelzung von Magie und Technologie auf Arcanis führt zur Entstehung einzigartiger Artefakte, die man durchaus als "Zauber-Waffen" oder "Magitech-Geräte" bezeichnen könnte. Durch spezielle und oft aufwendige Rituale ist es möglich, eine Portion magischer Energie in einen Gegenstand zu binden. Diese gebundene Magie kann dann bei Bedarf freigesetzt werden, was dem Gegenstand eine temporäre oder dauerhafte magische Eigenschaft verleiht. Dies könnte ein Schwert sein, das bei Aktivierung von Flammen umhüllt wird, oder eine Pistole, deren Kugeln mit einer magischen Lähmung versehen sind.

Allerdings sind Geräte, die mit Magie funktionieren, vergleichsweise selten und unterliegen einer wesentlichen Einschränkung: Sie können in der Regel nur zum eigenen Zweck des Besitzers genutzt werden. Der Grund hierfür liegt in der tiefen und persönlichen Bindung des Geburtskristalls an den Menschen. Da die Magie aus diesem persönlichen Kristall stammt, sind die daraus resultierenden magisch verstärkten Gegenstände ebenfalls an diese Bindung geknüpft. Ein Zauber-Schwert, das von einem Magier mit seiner eigenen Energie verzaubert wurde, wird für einen anderen Magier, der eine andere Verbindung zu seinem Kristall hat, nicht die gleiche Wirksamkeit oder überhaupt keine Funktion besitzen. Dies macht solche Magitech-Geräte zu hochgradig personalisierten Werkzeugen und erklärt ihre Seltenheit im alltäglichen Gebrauch, während sie in den Händen ihres rechtmäßigen Schöpfers zu beeindruckenden Instrumenten werden können.







%* **Grundlagen der Kristallmagie:**
%
%* **Magische Grundlagenfähigkeiten:** Welche einfachen magischen Effekte kann jeder beherrschen, der einen Kristall besitzt? (Z.B. kleine Lichter erzeugen, Gegenstände schweben lassen, einfache Reparaturen).
%
%Zu beginn beträgt die Kristallstufe 1. Hiermit kann jeder Zauber wirken, die ebenalls die Stufe besitzten. 
%
%* **Müdigkeit/Erschöpfung:** Wie wirkt sich die Magieanwendung auf den Anwender aus? (Z.B. Erschöpfungspunkte, temporärer Verlust der Kristallenergie).
%
%Jeder zauberer der Magie wirkt, verliert etwas von seinem Aether. Diese kann nur durch ausruhen erreichen. Mit Hilfe der Leylinien werden die Kristalle wieder mit Aether gefüllt.
%
%* **Einfache Zauber:** Eine kurze Liste von grundlegenden, oft genutzten "Alltagszaubern", die von jedem ohne spezielle Ausbildung gewirkt werden können.
%
%Es gibt keinen Basiszauber, die jeder von vornherein kann. Jeder Zauber muss erlernt werden. Hierfür ist eben jemand notwendig, der diesen Zauber beherrscht. Dies kann ein Lehrmeister an der Akademie sein, eine Hexe die allein im Wald wohnt, der verschrobene Druide, der ständig Kräuter sammelt oder eben einfach nur dein Elterneteil, das die beibringt was man machen muss um den Zauber zu wirken.
%
%* **Magische Interaktion mit Technologie:** Wie beeinflusst Magie die Steampunk-Technologie und umgekehrt?
%
%Die Magie selbst hat keine Einwirkung auf die Technologie. Sie können aber kombiniert werden. Daraus können dann faszinierende Dinge erschaffen werden.
%
%* Kann Magie Maschinen antreiben oder stören?
%
%Es gibt weitere Kristalle auf Arcanis, die Energie speichern können. Diese kann jeder entnehmen, der diese Energie einsetzten will. Mit dieser Energie können keine arkanen Formeln gewirkt werden.
%Die Wissenschaftler konnten die Energie für den Bau von Maschinen einsetzten. Diese Kristalle können durch den Einsatz von Maschinen auch aufgeladen werden. Wenn die Kristalle zu lange nicht aufgeladen werden, verlieren Sie diese Fähigkeit und zerfallen zu Staub. Je nach Energievorrat haben die Kristalle eine andere Farbe. Vn schwarz, was einen komplett leeren Kristall darstellt, bis zu einem gleißend weißen Kristall, der komplett gefüllt ist.
%Anhand der Farben wie bei einem Regenbogens kann der Füllstand abgelesen werden. folgend die Farben von Voll bis leer: Weiß, Rot, orange, gelb, grün, blau, indigo, violett, schwarz. Wenn der Kristall schearz ist, dann hat man einen Tag Zeit Ihn wieder zu befüllen. Geschieht dies nicht, dann zerfällt er zu staub.
%
%* Können Maschinen Magie verstärken oder kanalisieren?
%
%Maschinen sind nicht in der Lage Magie zu verstärken.
%
%* Gibt es "Zauber-Waffen" oder "Magitech-Geräte"?
%
%Durch spezielle Riten, kann Magie in einen Gegenstand gebunden werden, und diese kann dann bei Bedarf freigesetzt werden. Geräte die mit Magie funktionieren, sind möglich, aber selten, da Sie immer nur zum eigenen Zweck genutzt werden können. Der Grund hierfür ist, dass die Kristalle ja gebunden sind an einen Menschen.



\chapter{Magische Spezialisierungen und Traditionen}

* **Die Rolle der Magierklassen:** Erklärung, dass trotz der allgemeinen Magiefähigkeit spezialisierte Magieanwender existieren.
* **Die Ausbildung:** Wie erlangen diese Spezialisten ihr Wissen und ihre Fähigkeiten? (Gilden, Akademien, persönliche Mentoren, Studien uralter Schriften).
* **Unterscheidung von "Zauberern" und "Magiemeistern":** Wie unterscheiden sich die Fähigkeiten eines Laien von denen eines erfahrenen Magiers?
* **Spezialisierte Magier-Archetypen (Beispiele):**
* **Die Arkanisten (Magier):** Fokus auf theoretisches Wissen, komplexe Rituale, elementare Magie, Illusionen, Schutz- und Angriffszauber.
* Besondere Fähigkeiten/Merkmale (z.B. Fokus auf bestimmte Elemente, Beschwörungen von Entitäten).
* **Die Hexen (Elementare/Naturmagie):** Fokus auf Bindung zur Natur, Flüche, Segen, Heilung, Wettermanipulation, Beeinflussung von Pflanzen und Tieren.
* Besondere Fähigkeiten/Merkmale (z.B. Pakt mit Naturgeistern, Nutzung von Kräutern und Tränken).
* **Die Druiden (Wächter des Gleichgewichts):** Fokus auf Gleichgewicht und Harmonie zwischen Natur und Technologie, Gestaltwandel, Schutz des Lebens.
* Besondere Fähigkeiten/Merkmale (z.B. Kommunikation mit Tieren, Beherrschung natürlicher Kräfte).
* **Die Scharlatane/Illusionisten:** Fokus auf Täuschung, Verstellung, Beeinflussung des Geistes, Hypnose.
* Besondere Fähigkeiten/Merkmale (z.B. Meister der Verkleidung, Manipulation von Emotionen).
* **Die Kristall-Schmiede/Gravierer:** Fokus auf die Verstärkung und Modifikation von Kristallen, das Erschaffen von magischen Artefakten.
* Besondere Fähigkeiten/Merkmale (z.B. Einbringen von Zaubern in Gegenstände, Schaffung von magischen Rüstungen).
* **Zauberlisten pro Spezialisierung:** Eine detaillierte Liste von Zaubern, die jede Spezialisierung lernen und wirken kann, mit:
* **Name des Zaubers:**
* **Kategorie:** (z.B. Heilung, Beschwörung, Illusion, Angriff, Verteidigung).
* **Effekt:** Eine genaue Beschreibung, was der Zauber bewirkt.
* **Voraussetzungen:** (z.B. Stufe, Fertigkeitswert, Komponenten).
* **Kosten:** (z.B. Mana-Punkte, Erschöpfung, Materialkomponenten, Zeitaufwand).
* **Wirkungsbereich/Ziele:**
* **Dauer:**
* **Optional: Magische Formeln/Rituale:** Wie werden die Zauber gewirkt? (Gesten, verbale Komponenten, Fokusobjekte).

\chapter{Die Götter und ihre Magie}

* **Die Ära der Götter:** Eine historische Einordnung der Götter und ihrer Rolle in der Welt.
* Wann existierten sie?
* Was ist mit ihnen geschehen? (Verschwunden, schlafend, tot?).
* **Göttliche Magie:**
* **Die Essenz der göttlichen Magie:** Wie unterschied sich die Magie der Götter von der "normalen" Kristallmagie? War sie mächtiger, anders geartet, an Rituale gebunden?
* **Artefakte der Götter:** Gibt es noch Überreste ihrer Macht in der Welt (Artefakte, heilige Stätten, verlorene Wissen)?
* **Göttliche Intervention:** Können Götter (oder ihre Essenz) noch in die Welt eingreifen? (Z.B. durch Prophezeiungen, Visionen, direkte Manifestationen).
* **Aspekte der göttlichen Magie:** Gibt es bestimmte Magie-Schulen oder Effekte, die einzigartig für die Götter waren? (Z.B. Erschaffung von Leben, Manipulation der Zeit, Veränderung der Realität).
* **Göttlicher Einfluss auf Kristalle:** Hatten die Götter Einfluss auf die Ursprünge der Kristalle oder ihre Funktionsweise? (Z.B. haben sie die Kristalle den Menschen gegeben oder modifiziert?).

\chapter{Magische Phänomene und Gefahren}

* **Magische Anomalien:** Bereiche oder Ereignisse, in denen Magie unvorhersehbar oder besonders stark ist (z.B. Orte der alten Götter, magische Verwerfungen, spontane magische Ausbrüche).
* **Anti-Magie/Magie-Unterdrückung:** Gibt es Möglichkeiten, Magie zu blockieren oder zu unterdrücken? (Z.B. besondere Materialien, Rituale, Technologien).
* **Magische Krankheiten/Verfluchungen:** Wie können Magieanwender oder andere durch Magie geschädigt werden? (Z.B. Magie-Burnout, Kristall-Vergiftung, Flüche, Dämonische Besessenheit).
* **Verbotene Magie/Dunkle Künste:** Gibt es Magieformen, die als gefährlich oder illegal gelten? (Z.B. Nekromantie, Dämonologie, Gedankenkontrolle). Was sind die Konsequenzen der Ausübung solcher Magie?
* **Magische Wesen:** Gibt es Kreaturen, die von Natur aus magisch sind oder durch Magie erschaffen wurden? (Z.B. Elementare, Golems, arkane Konstrukte, Geister).

\chapter{Magische Regeln und Mechaniken}

* **Magiewirken im Detail:**
* **Wurfmechanismen:** Wie wird entschieden, ob ein Zauber gelingt? (Würfelwürfe, Fertigkeitswerte, Attributswerte).
* **Schwierigkeitsgrade:** Wie wird die Komplexität eines Zaubers berücksichtigt?
* **Komponenten:** Müssen Zauber mit verbalen, gestischen oder materiellen Komponenten gewirkt werden?
* **Konzentration:** Muss ein Zaubernder seine Konzentration aufrechterhalten, um einen Zauber zu wirken?
* **Modifikatoren für Zauber:**
* **Umgebung:** Wie beeinflusst die Umgebung (z.B. magische Orte, anti-magische Zonen, Wetter) die Zauberwirkung?
* **Ausrüstung:** Gibt es magische Gegenstände, die Zauber verstärken oder erleichtern?
* **Zustand des Zaubernden:** Wie beeinflusst Müdigkeit, Verletzung oder psychischer Zustand die Magiewirkung?
* **Zauber misslingen:** Was passiert, wenn ein Zauber misslingt? (Z.B. Rückschlag, ungewollte Effekte, Schaden für den Zaubernden).
* **Magische Forschung und Entwicklung:** Wie können Spieler neue Zauber erlernen oder bestehende modifizieren?
* **Magische Gegenstände und Artefakte:**
* Regeln für die Herstellung, Identifizierung und Nutzung von magischen Gegenständen.
* Listen von Beispielartefakten und deren Effekten.

\chapter{Der Geburtskristall im Detail (Erweiterte Regeln)}

* **Kristall-Typen:** Gibt es verschiedene Arten von Kristallen mit unterschiedlichen Eigenschaften oder Affinitäten? (Z.B. Elementar-Kristalle, Seelen-Kristalle).
* **Kristall-Entwicklung:** Können Kristalle mit der Zeit stärker werden oder sich verändern?
* **Kristall-Bindung:** Ist die Bindung an den Kristall reversibel? Was passiert, wenn der Kristall zerbricht oder gestohlen wird?
* **Kristall-Überladung/Überhitzung:** Was passiert bei übermäßiger Nutzung der Magie?
* **Kristall-Modifikationen:** Können Kristalle geschliffen, graviert oder mit anderen Materialien kombiniert werden, um ihre Kräfte zu verändern?





































\section{Wo kommt die Magie her?}

\chapter{Akademien der arkanen Formeln}

\chapter{Kristall}


\chapter{Zauber}





\rpgMakeMap[
image = Bilder/cover%,
%logo = img/logo,
%title = Basisregelwerk,
%subtitle = 
]


\end{document}
